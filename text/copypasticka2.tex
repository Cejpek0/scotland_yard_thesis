% Tento soubor nahraďte vlastním souborem s přílohami (nadpisy níže jsou pouze pro příklad)

% Pro kompilaci po částech (viz projekt.tex), nutno odkomentovat a upravit
%\documentclass[../projekt.tex]{subfiles}
%\begin{document}

% Umístění obsahu paměťového média do příloh je vhodné konzultovat s vedoucím
%\chapter{Obsah přiloženého paměťového média}

%\chapter{Manuál}

%\chapter{Konfigurační soubor}

%\chapter{RelaxNG Schéma konfiguračního souboru}

%\chapter{Plakát}


\chapter{Checklist} 
\label{checklist}
Tento checklist byl převzat ze šablony pro kvalifikační práce, která je k dispozici na blogu prof. Herouta \cite{Herout}, který s laskavým dovolením využil nápadu dr. Szökeho%
\footnote{\url{http://blog.igor.szoke.cz/2017/04/predstartovni-priprava-letu-neni.html}}. 

Velká bezpečnost letecké dopravy stojí z části na tom, že lidé kolem letadel mají \textbf{checklisty} na úplně každý, třeba rutinní a dobře zažitý, postup. Jako pilot strpí to, že bude trochu za blbce a opravdu tužtičkou do seznamu úkonů odškrtá dokonale zvládnuté akce, vytiskněte si a odškrtejte před odevzdáním diplomky i vy tento checklist a vyhněte se tak častým chybám, které by mohly mít až fatální následky na výsledné hodnocení Vaší práce.

\subsubsection*{Struktura}
\begin{checklist}
	\item Už ze samotných názvů a struktury kapitol je patrné, že bylo splněno zadání.
	\item V textu se nevyskytuje kapitola, která by měla méně než čtyři strany (kromě úvodu a závěru). Pokud ano, radil(a) jsem se o tom s vedoucím a ten to schválil.
\end{checklist}

\subsubsection*{Obrázky a grafy}
\begin{checklist}
	\item Všechny obrázky a tabulky byly zkontrolovány a jsou poblíž místa, odkud jsou z textu odkazovány, takže nebude problém je najít.
	\item Všechny obrázky a tabulky mají takový popisek, že celý obrázek dává smysl sám o~sobě, bez čtení dalšího textu. Vůbec nevadí, když má popisek několik řádků.
	\item Pokud je obrázek převzatý, tak je to v popisku zmíněno: \uv{Převzato z [X].}
	\item Písmenka ve všech obrázcích používají font podobné velikosti, jako je okolní text (ani výrazně větší, ani výrazně menší).
	\item Grafy a schémata jsou vektorově (tj. v PDF).
	\item Snímky obrazovky nepoužívají ztrátovou kompresi (jsou v PNG).
	\item Všechny obrázky jsou odkázány z textu.
	\item Grafy mají popsané osy (název osy, jednotky, hodnoty) a podle potřeby mřížku.
\end{checklist}

\subsubsection*{Rovnice}
\begin{checklist}
	\item Identifikátory a jejich indexy v rovnicích jsou jednopísmenné (kromě nečastých zvláštních případů jako $t_\mathrm{max}$).
	\item Rovnice jsou číslovány.
	\item Za (nebo vzácně před) rovnicí jsou vysvětleny všechny proměnné a funkce, které zatím vysvětleny nebyly.
\end{checklist}

\subsubsection*{Citace}
\begin{checklist}
    \item \textbf{Všechny použité zdroje jsou citovány.}
	\item Adresy URL odkazující na služby, projekty, zdroje, github apod. jsou odkazovány pomocí \verb|\footnote{\url{...}}|.
    \item Všechny citace používají správné typy.
	\item Citace mají autora, název, vydavatele (název konference), rok vydání.  Když některá nemá, je to dobře zdůvodněný zvláštní případ a vedoucí to odsouhlasil.
	\item Je-li ve zdrojových textech programu něco převzaté, je to tam řádně citováno v souladu s licencí.
	\item Je-li podstatná část zdrojových textů programu převzatá, je toto zmíněno v textu práce a je citován zdroj.
\end{checklist}

\subsubsection*{Typografie}
\begin{checklist}
	\item Žádný řádek nepřetéká přes pravý okraj.
	\item Na konci řádku nikde není jednopísmenná předložka (spraví to nedělitelná mezera $\sim$).
	\item Číslo obrázku, tabulky, rovnice, citace není nikde první na novém řádku (spraví to nedělitelná mezera $\sim$).
	\item Před číselným odkazem na poznámku pod čarou nikde není mezera (to jest vždy takto\footnote{příklad poznámky pod čarou}, nikoliv takto \footnote{jiný příklad poznámky pod čarou}).
\end{checklist}

\subsubsection*{Jazyk}
\begin{checklist}
    \item Použil jsem kontrolu pravopisu a v textu nikde nejsou překlepy.
	\item Nechal jsem si text přečíst od (alespoň) jednoho dalšího člověka, který umí dobře česky / anglicky / slovensky.
	\item V práci psané česky nebo slovensky abstrakt zkontroloval někdo, kdo umí opravdu dobře anglicky.
	\item V textu se nikde nepoužívá druhá mluvnická osoba (vy/ty).
	\item Když se v textu vyskytuje první mluvnická osoba (já, my), vždy se popisuje subjektivní záležitost (\textit{rozhodl jsem se}, \textit{navrhl jsem}, \textit{zaměřil jsem se na}, \textit{zjistil jsem} apod.).
	\item V textu se nikde nepoužívají hovorové výrazy.
	\item V českém či slovenském textu se zbytečně nepoužívají anglické výrazy, které mají ustálené české překlady. Např. slovo \textit{defaultní} se nahradí např. slovem \textit{implicitní} nebo \textit{výchozí}.
\end{checklist}

\subsubsection*{Výsledek na datovém médiu, tj. software}
\begin{checklist}
	\item Mám připravené nepřepisovatelné datové médium 
      \begin{itemize}
	  		\item CD-R,
            \item DVD-R,
            \item DVD+R ve formátu ISO9660 (s rozšířením RockRidge a/nebo Jolliet) nebo UDF,
            \item paměťová karta SD (Secure Digital) ve formátu FAT32 nebo exFAT s nastavenou ochranou proti přepisu.
      \end{itemize}
	\item Pokud je výsledek online (služba, aplikace, \dots), URL je viditelně v úvodu a závěru, aby bylo jasné, kde výsledek hledat.
	\item Na médiu nechybí povinné: 
    	\begin{itemize}
    		\item zdrojové kódy (např. Matlab, C/C++, Python, \dots)
            \item knihovny potřebné pro překlad,
            \item přeložené řešení,
            \item PDF s technickou zprávou (je-li pro tisk 2. verze, tak obě),
            \item zdrojový kód zprávy (\LaTeX), 
    	\end{itemize}
        a případně volitelně po dohodě s vedoucím práce
		\begin{itemize}
			\item relevantní (např. testovací) data, 
            \item demonstrační video,
            \item PDF plakátku,
            \item \dots
		\end{itemize}        
	\item Zdrojové kódy jsou refaktorovány, komentovány a označeny hlavičkou s autorstvím, takže se v nich snadno vyzná i někdo další, než sám autor.
    \item Jakákoliv převzatá část zdrojového kódu je řádně citována -- tedy označena úvodním a v případě převzetí více řádků i ukončovacím komentářem. Komentář obsahuje vše, co vyžaduje licence uvedená na webu (vždy je nutné se ji pokusit najít -- např. Stack Overflow\footnote{\url{https://stackoverflow.blog/2009/06/25/attribution-required/}} má striktní pravidla pro citace).
\end{checklist}

\subsubsection*{Odevzdání}

\begin{checklist}
\item Chci práci (na max. 3 roky) utajit? Pokud ano, nejpozději měsíc před termínem odevzdání práce si podám žádost (v IS), ke které přiložím případné stanovisko firmy, jejíž duševní vlastnictví je třeba chránit.
\item Mám splněný minimální počet normostran textu (lze spočítat pomocí Makefile a~odhadem přičíst obrázky). Pokud jsem těsně pod minimem, konzultoval(a) jsem to s~vedoucím.
\item Pokud chci tisknout oboustranně, konzultoval(a) jsem to s~vedoucím a mám správně nastavenou šablonu. Kapitoly začínají na liché stránce.
\item Technickou zprávu mám v deskách z knihařství (min. 1 výtisk, při utajení oba).
\item Za titulním listem práce je zadání (tzn. mám jej stažené z IS a vložené do šablony).
\item V IS jsou abstrakty a klíčová slova.
  \begin{itemize}
    \item V abstraktu a klíčových slovech v IS nejsou zkopírované vlnky pro nezlomitelné mezery.
  \end{itemize}      
\item V IS je PDF práce (s klikatelnými odkazy).
\item Oba výtisky práce jsou podepsané.
\item V jednom (při utajení obou) výtisku práce je paměťové médium, na kterém je fixkou napsaný login (fixku na CD lze zapůjčit v knihovně, na Studijním oddělení nebo až při odevzdání).
\end{checklist}


\chapter{\LaTeX pro začátečníky}
\label{latex}

V této kapitole jsou uvedeny některé často využívané balíčky a příkazy pro \LaTeX{}, které mohou být při tvorbě práce potřeba.

\subsection*{Užitečné balíčky}

Studenti při sazbě textu často řeší stejné problémy. Některé z nich lze vyřešit následujícími balíčky pro \LaTeX:

\begin{itemize}
  \item \verb|amsmath| -- rozšířené možnosti sazby rovnic,
  \item \verb|float, afterpage, placeins| -- úprava umístění obrázků/tabulek (specifikátor \texttt{H}),
  \item \verb|fancyvrb, alltt| -- úpravy vlastností prostředí Verbatim, 
  \item \verb|makecell| -- rozšíření možností tabulek,
  \item \verb|pdflscape, rotating| -- natočení stránky o 90 stupňů (pro obrázek či tabulku),
  \item \verb|hyphenat| -- úpravy dělení slov,
  \item \verb|picture, epic, eepic| -- přímé kreslení obrázků.
\end{itemize}

Některé balíčky jsou využity přímo v šabloně (v dolní části souboru \texttt{fitthesis.cls}). Nahlédnutí do jejich dokumentace může být rovněž velmi užitečné.

Sloupec tabulky zarovnaný vlevo s pevnou šířkou je v šabloně definovaný \uv{L} (používá se jako \uv{p}).

Pro odkazování v rámci textu použijte příkaz \verb|\ref{navesti}|. Podle umístění návěští se bude jednat o~číslo kapitoly, podkapitoly, obrázku, tabulky nebo podobného číslovaného prvku). Pokud chcete odkázat stránku práce, použijte příkaz \verb|pageref{navesti}|. Pro citaci literárního odkazu \verb|\cite{identifikator}|. Pro odkazy na rovnice lze použít příkaz \verb|\eqref{navesti}|.

Znak \,--\, (pomlčka) se V \LaTeX u vkládá jako dvě mínus za sebou: -{}-.

\subsection*{Často využívané příkazy pro \LaTeX{}}
\label{sec:Fragments}

Doporučuji nahlédnout do zdrojového textu této podkapitoly a podívat se, jak jsou následující ukázky vysázeny. Ve zdrojovém textu jsou i pomocné komentáře.

% Sloupec zarovnaný vlevo s pevnou šířkou je v šabloně definovaný "L" (používá se jako p)

Příklad tabulky:
\begin{table}[H]
	\vskip6pt
	\caption{Tabulka hodnocení} 
    \vskip6pt
	\centering
	\begin{tabular}{llr}
		\toprule
		\multicolumn{2}{c}{Jméno} \\
		\cmidrule(r){1-2}
		Jméno & Příjmení & Hodnocení \\
		\midrule
		Jan & Novák & $7.5$ \\
		Petr & Novák & $2$ \\
		\bottomrule
	\end{tabular}
	\label{tab:ExampleTable}
\end{table}

% Ohraničení lze upravit dle potřeby:
% http://latex-community.org/forum/viewtopic.php?f=45&t=24323
% http://tex.stackexchange.com/questions/58163/problem-with-multirow-and-table-cell-borders
% http://tex.stackexchange.com/questions/79369/formatting-table-border-and-text-alignment-in-latex-table

\noindent Příklad rovnice:
\begin{equation}
	\cos^3 \theta =\frac{1}{4}\cos\theta+\frac{3}{4}\cos 3\theta
	\label{eq:rovnice2}
\end{equation}
a dvou horizontálně zarovnaných rovnic: % znak & řídí zarovnání
\begin{align} 
    \label{eq:soustava}
	3x &= 6y + 12 \\
	x &= 2y + 4 
\end{align}

Pokud je třeba rovnici citovat v textu, lze použít příkaz \verb|\eqref|. Například na rovnici výše lze odkázat~\eqref{eq:rovnice2}. Pokud chcete srovnat číslo rovnic u soustavy, lze použít prostředí \texttt{split}:
\begin{equation} \label{eq:soustavaSrovnana}
\begin{split}
	3x &= 6y + 12 \\
	x &= 2y + 4
\end{split}
\end{equation}

Matematické symboly ($\alpha$) a výrazy lze umístit i do textu $\cos\pi=-1$ a mohou být i~v~poznámce pod čarou%
\footnote{Vzorec v poznámce pod čarou: $\cos\pi=-1$}.

Obrázek~\ref{sirokyObrazek} ukazuje široký obrázek složený z více menších obrázků. Klasický rastrový obrázek se vkládá tak, jak je vidět na obrázku \ref{keepCalm}.

% Využití \begin{figure*} způsobí, že obrázek zabere celou šířku stránky. Takový obrázek dříve mohl být pouze na začátku stránky, případně na konci s využitím balíčku dblfloatfix (případné [h] se ignorovalo a [H] obrázek odstraní). Nové verze LaTeXu už umí i [h].
\begin{figure*}[h]\centering
  \centering
  \includegraphics[width=\linewidth,height=1.7in]{obrazky-figures/placeholder.pdf}\\[1pt]
  \includegraphics[width=0.24\linewidth]{obrazky-figures/placeholder.pdf}\hfill
  \includegraphics[width=0.24\linewidth]{obrazky-figures/placeholder.pdf}\hfill
  \includegraphics[width=0.24\linewidth]{obrazky-figures/placeholder.pdf}\hfill
  \includegraphics[width=0.24\linewidth]{obrazky-figures/placeholder.pdf}
  \caption{\textbf{Široký obrázek.} Obrázek může být složen z více menších obrázků. Chcete-li se na tyto dílčí obrázky odkazovat z textu, využijte balíček \texttt{subcaption}.}
  \label{sirokyObrazek}
\end{figure*}

% Odkomentujte pro přepnutí na formát A3 na šířku
% \eject \pdfpagewidth=420mm

\begin{figure}[hbt]
	\centering
	\includegraphics[width=0.3\textwidth]{obrazky-figures/keep-calm.png}
	\caption{Dobrý text je špatným textem, který byl několikrát přepsán. Nebojte se prostě něčím začít.}
	\label{keepCalm}
\end{figure}

Někdy je potřeba do příloh umístit diagram, který se nevejde na stránku formátu A4. Pak je možné vložit jednu stránku formátu A3 a do práce ji poskládat (tzv. skládání do~Z, kdy se vytvoří dva sklady -- lícem dolů a lícem nahoru, angl. Engineering fold -- existuje i~anglický pojem Z-fold, ale při tom by byl problém s vazbou). Přepnutí se provádí následovně: \texttt{\textbackslash{}eject \textbackslash{}pdfpagewidth=420mm} (pro přepnutí zpět pak 210mm).

Další často využívané příkazy naleznete ve zdrojovém textu ukázkového obsahu této šablony.

% Odkomentujte pro přepnutí zpět na A4
% \eject \pdfpagewidth=210mm


\newcommand{\odradkovani}{\\[0.3em]}

\chapter{Příklady bibliografických citací}
\label{priloha-priklady-citaci}
Styl czplain vychází ze stylu vytvořeného v rámci práce pana Pyšného \cite{Pysny}. Obsahuje sadu podporovaných typů citací s konkrétními příklady bibliografických citací. 

Na následujících stránkách přílohy jsou uvedeny příklady, jenž znázorňují bibliografické citace následujících publikací a~jejich částí:
\begin{itemize}
   \item článku v seriálové publikaci (časopisu) (str. \pageref{pr-casopis-clanek}),
   \item monografické publikace (str. \pageref{pr-monografie}),
   \item sborníku (str. \pageref{pr-sbornik}),
   \item článku ve sborníku nebo kapitoly v knize (str. \pageref{pr-kapitola}),
   \item manuálu, dokumentace, technické zprávy a nepublikovaných materiálů (str. \pageref{pr-manual}),
   \item akademické práce (str. \pageref{pr-thesis}),
   \item webové stránky (str. \pageref{pr-webpage}),
   \item a webové domény (str. \pageref{pr-website}).
\end{itemize}

\noindent Položky jsou označený barevně podle povinnosti:
\begin{itemize}
    \item prvek je dle normy povinný
    \item \textcolor{blue}{prvek, který je dle normy volitelný}
    \item \textcolor{magenta}{prvek, který je dle normy povinný pro online informační zdroje}
    \item \textcolor{red}{prvek, který není předepsán normou a je v bibliografickém stylu v šabloně volitelný}
\end{itemize}
Povinné položky se uvádí pouze pokud existují.

\newpage
\noindent V souboru s bibliografií se záznamy uvádí následujícím způsobem:
\begin{verbatim}
@Article{Doe:2020,
   author               = "Doe, John",
   title                = "Jak citovat",
   subtitle             = "Citace článku",
   journal              = "Seriál o tvorbě prací",
   journalsubtitle      = "Formální náležitosti",
   howpublished         = "online",
   address              = "Brno",
   publisher            = "Fakulta informačních technologií VUT v Brně",
   contributory         = "Přeložil Jan NOVÁK",
   edition              = "1",
   version              = "verze 1.0",
   month                = 2,
   year                 = "2020",
   revised              = "revidováno 12. 2. 2020",
   volume               = "4",
   number               = "24",
   pages                = "8--21",
   cited                = "2020-02-12",
   doi                  = "10.1000/BC1.0",
   issn                 = "1234-5678",
   note                 = "Toto je zcela vymyšlená citace",
   url                  = "https://merlin.fit.vutbr.cz"
}
\end{verbatim}

Citace jsou seřazeny podle abecedy. Řazení jmen s písmeny s diakritikou můžeme ovlivnit prvkem \texttt{key}, jehož  hodnotu nastavíme na příjmení bez diakritiky. Pokud není vyplněn autor, citace se řadí na začátek seznamu, což není vhodné. Řazení v tomto případě můžeme taktéž ovlivnit vhodně nastaveným prvkem key.

\medskip
\medskip
\noindent \textbf{Příklad}:
\begin{verbatim}
   @Article{Cech:2020:Citace,
	   author               = "Čech, Jan",
	   key                  = "Cech",
	   ... 
\end{verbatim}


%-------------------------------------------------------------------------------
\newpage
\section*{Článek v seriálové publikaci -- @Article}
\label{pr-casopis-clanek}
\noindent \textbf{Položky záznamu}

\medskip

\begin{tabularx}{0.95\linewidth}{>{\raggedright\arraybackslash}X X >{\raggedright\arraybackslash}X}
    Prvek & Zápis v BibTeXu & Příklad \\\hline
    Tvůrce & author & Doe, John\\
    Název příspěvku & title & Jak citovat\\
    \textcolor{blue}{Vedlejší název} & \textcolor{blue}{subtitle} & \textcolor{blue}{Citace článku}\\
    Název seriálové publikace & journal & Seriál o tvorbě prací\\
    \textcolor{blue}{Vedlejší názvy seriálu} & \textcolor{blue}{journalsubtitle} & \textcolor{blue}{Formální náležitosti}\\
    \textcolor{magenta}{Typ nosiče} & \textcolor{magenta}{howpublished} & \textcolor{magenta}{online}\\
    Vydání & edition & 1\\
    Verze & version & verze 1.0\\
    \textcolor{blue}{Další tvůrce} & \textcolor{blue}{contributory} & \textcolor{blue}{Přeložil Jan NOVÁK}\\
    Místo vydání & address & Brno\\
    Nakladatel & publisher & Fakulta informačních technologií VUT v Brně\\
    Měsíc & month & 2\\
    Rok & year & 2020\\
    Svazek & volume & 4\\
    Číslo & number & 24\\
    Rozsah příspěvku & pages & 8-21\\
    Revize & revised & revidováno 12. 2. 2020\\
    \textcolor{magenta}{Datum citování} & \textcolor{magenta}{cited} & \textcolor{magenta}{2020-02-12}\\
    Název edice & series & Návody k tvorbě prací\\
    Číslo edice & editionnumber & 42\\
    \textcolor{magenta}{Identifikátor digitálního obsahu} & \textcolor{magenta}{doi} & \textcolor{magenta}{10.1000/BC1.0}\\
    Standardní číslo  & issn & 1234-5678\\
    \textcolor{red}{Poznámky} & \textcolor{red}{note} & \textcolor{red}{Toto je zcela vymyšlená citace}\\
    \textcolor{magenta}{Dostupnost a přístup} & \textcolor{magenta}{url} & \textcolor{magenta}{https://merlin.fit.vutbr.cz}
\end{tabularx}

\bigskip

\noindent \textbf{Bibliografická citace}

\medskip

\noindent \textsc{Doe}, J. Jak Citovat: Citace článku. \textit{Seriál o tvorbě prací: Formální náležitosti} [online]. 1.~vyd., verze 1.0. Přeložil Jan NOVÁK. Brno: Fakulta informačních technologií VUT v~Brně. Únor 2020, sv. 4, č. 24, s. 8–21, revidováno 12. 2. 2020, [cit. 2020-02-12]. Návody k~tvorbě prací, č. 42. DOI: 10.1000/BC1.0. ISSN 1234-5678. Toto je zcela vymyšlená citace. Dostupné z: \url{https://merlin.fit.vutbr.cz}

%-------------------------------------------------------------------------------
\newpage
\section*{Monografická publikace -- @Book, @Booklet (kniha, brožura)}
\label{pr-monografie}
\noindent \textbf{Položky záznamu}

\medskip

\begin{tabularx}{0.95\linewidth}{X X >{\raggedright\arraybackslash}X}
    Prvek & Zápis v BibTeXu & Příklad\\\hline
    Tvůrce & author & John von Doe\\
    Titul & title & Jak citovat\\
    \textcolor{blue}{Vedlejší názvy} & \textcolor{blue}{subtitle} & \textcolor{blue}{Citace monografické publikace}\\
    \textcolor{magenta}{Typ nosiče} & \textcolor{magenta}{howpublished} & \textcolor{magenta}{online}\\
    Vydání & edition & 1\\
    \textcolor{blue}{Další tvůrce} & \textcolor{blue}{contributory} & \textcolor{blue}{Přeložil Jan NOVÁK}\\
    Místo vydání & address & Brno\\
    Nakladatel & publisher & Fakulta informačních technologií VUT v Brně\\
    Měsíc vydání & month & 2\\
    Rok vydání & year & 2020\\
    Revize & revision & revidováno 12. 2. 2020\\
    \textcolor{magenta}{Datum citování} & \textcolor{magenta}{cited} & \textcolor{magenta}{2020-02-12}\\
    \textcolor{red}{Rozsah} & \textcolor{red}{pages} & \textcolor{red}{220}\\
    Edice & series & Návody k tvorbě prací\\
    Číslo edice & editionnumber & 2\\
    Standardní číslo & isbn & 01-234-5678-9\\
    \textcolor{red}{Poznámky} & \textcolor{red}{note} & \textcolor{red}{Toto je zcela vymyšlená citace}\\
    \textcolor{magenta}{Dostupnost a přístup} & \textcolor{magenta}{url} & \textcolor{magenta}{https://merlin.fit.vutbr.cz}\\
\end{tabularx}

\bigskip

\noindent \textbf{Bibliografická citace}

\medskip

\noindent \textsc{von Doe}, J. \textit{Jak citovat: Citace monografické publikace} [online] . 1. vyd. Přeložil Jan NOVÁK.
Brno:Fakulta informačních technologií VUT v Brně, únor 2020, revidováno 12. 2. 2020 [cit. 2020-02-12]. 220 s. Návody k tvorbě prací, č. 2. ISBN 01-234-5678-9. Toto je zcela vymyšlená citace. Dostupné z: \url{https://merlin.fit.vutbr.cz}
%-------------------------------------------------------------------------------
\newpage
\section*{Sborník -- @Proceedings}
\label{pr-sbornik}
\noindent \textbf{Položky záznamu}

\medskip

\begin{tabularx}{0.95\linewidth}{>{\raggedright\arraybackslash}X X >{\raggedright\arraybackslash}X}
    Prvek & Zápis v BibTeXu & Příklad\\\hline
    \textcolor{red}{Tvůrce*} & \textcolor{red}{author} & \textcolor{red}{Čechmánek, Jan}\\
    \textcolor{red}{Editor*} & \textcolor{red}{editor} & \textcolor{red}{Čechmánek, Jan}\\
    Titul & title & Jak citovat\\
    \textcolor{blue}{Vedlejší názvy} & \textcolor{blue}{subtitle} & \textcolor{blue}{Citace monografické publikace}\\
    \textcolor{magenta}{Typ nosiče} & \textcolor{magenta}{howpublished} & \textcolor{magenta}{online}\\
    Vydání & edition & 1\\
    \textcolor{blue}{Další tvůrce} & \textcolor{blue}{contributory} & \textcolor{blue}{Přeložil Jan NOVÁK}\\
    Místo vydání & address & Brno\\
    Nakladatel & publisher & Fakulta informačních technologií VUT v Brně\\
    Měsíc vydání & month & 2\\
    Rok vydání & year & 2020\\
    Svazek & volume & 4\\
    Číslo svazku & number & 24\\
    Rozsah příspěvku & pages & 8-21\\
    \textcolor{magenta}{Revize} & \textcolor{magenta}{revised} & \textcolor{magenta}{revidováno 12. 2. 2020}\\
    \textcolor{magenta}{Datum citování} & \textcolor{magenta}{cited} & \textcolor{magenta}{2020-02-12}\\
    Edice & series & Návody k tvorbě prací\\
    Číslo edice & editionnumber & 2\\
    \textcolor{magenta}{Identifikátor digitálního objektu} & \textcolor{magenta}{doi} & \textcolor{magenta}{10.1000/BC1.0}\\
    Standardní číslo & isbn nebo issn & 01-234-5678-9\\
    \textcolor{red}{Poznámky} & \textcolor{red}{note} & \textcolor{red}{Toto je zcela vymyšlná citace}\\
    \textcolor{magenta}{Dostupnost a přístup} & \textcolor{magenta}{url} & \textcolor{magenta}{https://merlin.fit.vutbr.cz}
\end{tabularx}

*Uvádí se buď autor, nebo editor.

\bigskip

\noindent \textbf{Bibliografická citace}

\medskip

\noindent \textsc{Čechmánek}, J. \textit{Jak citovat: Citace sborníku} [online]. 1. vyd. Přeložil Jan NOVÁK.
Brno: Fakulta informačních technologií VUT v Brně, únor 2020, sv. 4, č. 24, s. 8–21, revidováno 12. 2. 2020 [cit. 2020-02-12]. Návody k tvorbě prací, č. 2. DOI: 10.1000/BC1.0. ISBN 01-234-5678-9. Toto je zcela vymyšlená citace. Dostupné z: \url{https://merlin.fit.vutbr.cz}
%-------------------------------------------------------------------------------
\newpage
\section*{Článek ve sborníku nebo kapitola v knize -- @InProceedings, @InCollection, @Conference, @InBook}
\label{pr-kapitola}
\noindent \textbf{Položky záznamu}

\medskip

\begin{tabularx}{0.95\linewidth}{X X >{\raggedright\arraybackslash}X}
    Prvek & Zápis v BibTeXu & Příklad\\\hline
    Tvůrce & author & John von Doe\\
    Název příspěvku & title & Jak citovat\\
    \textcolor{blue}{Vedlejší názvy} & \textcolor{blue}{subtitle} & \textcolor{blue}{Citace článku}\\
    Jméno tvůrce mateřského dokumentu & editor nebo organisation & Smith, Peter\\
    Název mateřského dokumentu & booktitle & Sborník konference o~tvorbě prací\\
    \textcolor{blue}{Vedlejší názvy mateřského dokumentu} & \textcolor{blue}{booksubtitle} & \textcolor{blue}{Formální náležitosti}\\
    \textcolor{magenta}{Typ nosiče} & \textcolor{magenta}{howpublished} & \textcolor{magenta}{online}\\
    Vydání & edition & 1\\
    Verze & version & verze 1.0\\
    \textcolor{blue}{Další původce mateřského dokumentu} & \textcolor{blue}{contributory} & \textcolor{blue}{Přeložil Jan NOVÁK}\\
    Místo vydání & address & Brno\\
    Nakladatel & publisher & Fakulta informačních technologií VUT v Brně\\
    Měsíc & month & 2\\
    Rok & year & 2020\\
    Svazek & volume & 4\\
    Číslo svazku & number & 24\\
    \textcolor{blue}{Kapitola} & \textcolor{blue}{chapter} & \textcolor{blue}{5}\\
    Rozsah příspěvku & pages & 8-21\\
    Revize & revised & revidováno 12. 2. 2020\\
    \textcolor{magenta}{Datum citování} & \textcolor{magenta}{cited} & \textcolor{magenta}{2020-02-12}\\
    Edice & series & Návody k tvorbě prací\\
    Číslo edice & editionnumber & 2\\
    Standardní číslo & isbn nebo issn & 1234-5678\\
    \textcolor{red}{Poznámky} & \textcolor{red}{note} & \textcolor{red}{Toto je zcela vymyšlená citace}\\
    \textcolor{magenta}{Dostupnost a přístup} & \textcolor{magenta}{url} & \textcolor{magenta}{https://merlin.fit.vutbr.cz}\\
\end{tabularx}

\bigskip

\noindent \textbf{Bibliografická citace}\\
\textsc{Doe}, J. Jak citovat: Citace článku.
In: \textsc{Smith}, P., ed. \textit{Sborník konference o tvorbě prací: Formální náležitosti} [online]. 1. vyd., verze 1.0. Přeložil Jan NOVÁK. Brno: Fakulta informačních technologií VUT v Brně, únor 2020, sv. 4, č. 24, kap. 5, s. 8–21, revidováno 12. 2. 2020 [cit. 2020-02-12]. Návody k tvorbě prací, č. 2. ISSN 1234-5678. Toto je zcela vymyšlená citace. Dostupné z: \url{https://merlin.fit.vutbr.cz}
%-------------------------------------------------------------------------------
\newpage
\section*{Manuál, dokumentace, technická zpráva a nepublikované materiály -- @Manual, @TechReport, @Unpublished}
\label{pr-manual}
\noindent \textbf{Položky záznamu}

\medskip

\begin{tabularx}{0.95\linewidth}{X X >{\raggedright\arraybackslash}X}
    Prvek & Zápis v BibTeXu & Příklad\\\hline
    Tvůrce (osoba nebo organizace) & author & Fakulta informačních technologií VUT v Brně\\
    Titul & title & Manuál k tvorbě prací\\
    \textcolor{blue}{Vedlejší názvy} & \textcolor{blue}{subtitle} & \textcolor{blue}{Citace manuálu}\\
    \textcolor{magenta}{Typ nosiče} & \textcolor{magenta}{howpublished} & \textcolor{magenta}{online}\\
    \textcolor{red}{Typ dokumantu} & \textcolor{red}{type} & \textcolor{red}{Uživatelský manuál}\\
    \textcolor{red}{Číslo dokumentu} & \textcolor{red}{number} & \textcolor{red}{3}\\
    Vydání & edition & 1\\
    \textcolor{blue}{Další tvůrce} & \textcolor{blue}{contributory} & \textcolor{blue}{Editoval Jan NOVÁK}\\
    Místo vydání & address & Brno\\
    Organizace nebo instituce & organization nebo institution & Fakulta informačních technologií VUT v Brně\\
    Měsíc vydání & month & 2\\
    Rok vydání & year & 2020\\
    Revize & revised & revidováno 12. 2. 2020\\
    \textcolor{magenta}{Datum citování} & \textcolor{magenta}{cited} & \textcolor{magenta}{2020-02-12}\\
    \textcolor{red}{Rozsah} & \textcolor{red}{pages} & \textcolor{red}{220}\\   
    \textcolor{red}{Poznámky} & \textcolor{red}{note} & \textcolor{red}{Toto je zcela vymyšlená citace}\\
    \textcolor{magenta}{Dostupnost a přístup} & \textcolor{magenta}{url} & \textcolor{magenta}{https://merlin.fit.vutbr.cz}\\
\end{tabularx}

\bigskip

\noindent \textbf{Bibliografická citace}

\medskip

\noindent \textsc{Fakulta informačních technologií VUT v Brně}. \textit{Manuál k tvorbě prací: Citace manuálu} [online]. Uživatelský manuál 3, 1. vyd. Editoval Jan NOVÁK.
Brno: Fakulta informačních technologií VUT v Brně, únor 2020, revidováno 12. 2. 2020 [cit. 2020-02-12]. 220 s. Toto je zcela vymyšlená citace. Dostupné z: \url{https://merlin.fit.vutbr.cz}
%-------------------------------------------------------------------------------
\newpage
\section*{Akademická práce -- @BachelorsThesis, @MastersThesis, \\@PhdThesis, @Thesis}
\label{pr-thesis}
\noindent \textbf{Položky záznamu}

\medskip

\begin{tabularx}{0.95\linewidth}{X X >{\raggedright\arraybackslash}X}
    Prvek & Zápis v BibTeXu & Příklad\\\hline
    Tvůrce & author & Fakulta informačních technologií VUT v Brně\\
    Titul & title & BiBTeX styl pro ČSN ISO 690 a ČSN ISO 690-2\\
    \textcolor{blue}{Vedlejší názvy} & \textcolor{blue}{subtitle} & \\
    \textcolor{magenta}{Typ nosiče} & \textcolor{magenta}{howpublished} & \textcolor{magenta}{online}\\
    \textcolor{red}{Typ dokumantu} & \textcolor{red}{type} & \textcolor{red}{Diplomová práce}\\
    Místo vydání & address nebo location & Brno\\
    Škola & school & Vysoké učení technické v~Brně, Fakulta informačních technologií\\
    Rok vydání & year & 2020\\
    \textcolor{magenta}{Datum citování} & \textcolor{magenta}{cited} & \textcolor{magenta}{2020-02-12}\\
    \textcolor{red}{Rozsah} & \textcolor{red}{pages} & \textcolor{red}{220}\\
    \textcolor{red}{Rozsah příloh} & \textcolor{red}{inserts} & \textcolor{red}{20}\\
    Standardní číslo & isbn & 01-234-5678-9\\
    \textcolor{red}{Vedoucí práce} & \textcolor{red}{Supervisor} & \textcolor{red}{Dytrych, Jaroslav}\\
    \textcolor{red}{Poznámky} & \textcolor{red}{note} & \textcolor{red}{Toto je zcela vymyšlená citace}\\
    \textcolor{magenta}{Dostupnost a přístup} & \textcolor{magenta}{url} & \textcolor{magenta}{https://www.fit.vut.cz/\-study/theses}\\
\end{tabularx}

\bigskip

\noindent \textbf{Bibliografická citace}

\medskip

\noindent \textsc{Novák}, J. \textit{BiBTeX styl pro ČSN ISO 690 a ČSN ISO 690-2} [online]. Brno, CZ, 2020. [cit. 2020-02-12]. 80 s., 20. s. příl. Diplomová práce. Vysoké učení technické v Brně, Fakulta informačních technologií. ISBN 01-2345-678-9. Vedoucí práce \textsc{Dytrych}, J. Toto je zcela vymyšlená citace. Dostupné z: \url{https://www.fit.vut.cz/study/theses}
%-------------------------------------------------------------------------------
\newpage
\section*{Webová stránka -- @Webpage}
\label{pr-webpage}
\noindent \textbf{Položky záznamu}

\medskip

\begin{tabularx}{0.95\linewidth}{>{\raggedright\arraybackslash}X X >{\raggedright\arraybackslash}X}
    Prvek & Zápis v BibTeXu & Příklad\\\hline
    Tvůrce & author & Nováková, Jana\\
    Název příspěvku & secondarytitle & Citace příspěvku\\
    Název stránky & title & Web tvorby prací\\
    \textcolor{blue}{Vedlejší název stránky}  &  \textcolor{blue}{subtitle} & \\
    \textcolor{magenta}{Typ nosiče} & \textcolor{magenta}{howpublished} & \textcolor{magenta}{online}\\
    \textcolor{blue}{Další tvůrce} & \textcolor{blue}{contributory} & \textcolor{blue}{Editoval Jan NOVÁK}\\
    \textcolor{red}{Verze} & \textcolor{red}{version} & \textcolor{red}{Verze 1.0}\\
    \textcolor{red}{Místo vydání} & \textcolor{red}{address} & \textcolor{red}{Brno}\\
    \textcolor{red}{Vydavatel} & \textcolor{red}{publisher} & \textcolor{red}{Fakulta informačních technologií VUT v Brně}\\
    Den & day & 12\\
    Měsíc vydání & month & 2\\
    Rok vydání & year & 2020\\
    \textcolor{blue}{Čas publikování} & \textcolor{blue}{time} & \textcolor{blue}{14:00}\\
    Revize & revised & Revidováno 12. 2. 2020\\
    \textcolor{magenta}{Identifikátor digitálního objektu} & \textcolor{magenta}{doi} & \textcolor{magenta}{10.1000/BC1.0}\\
    Standardní číslo & issn & 1234-5678\\
    \textcolor{red}{Poznámky} & \textcolor{red}{note} & \textcolor{red}{Toto je zcela vymyšlená citace}\\
    Dostupnost a přístup & url & https://merlin.fit.vutbr.cz\\
    Cesta & path & Domů; Umění; Umění citace
\end{tabularx}

\bigskip

\noindent \textbf{Bibliografická citace}

\medskip

\noindent \textsc{Nováková}, J. Citace příspěvku. \textit{Web tvorby prací} [online]. Editoval Jan NOVÁK. Verze 1.0. Brno: Fakulta informačních technologií VUT v Brně, 2. března 1998 14:10. Revidováno 12. 2. 2020 [cit. 2020-02-12]. DOI: 10.1000/BC1.0. ISSN 1234-5678. Toto je zcela vymyšlená citace. Dostupné z: \url{https://merlin.fit.vutbr.cz} Path: Domů; Umění; Umění citace.
%-------------------------------------------------------------------------------
\newpage
\section*{Webová doména -- @Website}
\label{pr-website}
\noindent \textbf{Položky záznamu}

\medskip

\begin{tabularx}{0.95\linewidth}{>{\raggedright\arraybackslash}X X >{\raggedright\arraybackslash}X}
    Prvek & Zápis v BibTeXu & Příklad\\\hline
    Tvůrce (osoba nebo organizace) & author & Nováková, Jana\\
    Název webu & title & Web tvorby prací\\
    \textcolor{blue}{Vedlejší název webu} &  \textcolor{blue}{subtitle} & \\
    \textcolor{magenta}{Typ nosiče} & \textcolor{magenta}{howpublished} & \textcolor{magenta}{online}\\
    \textcolor{blue}{Další tvůrce} & \textcolor{blue}{contributory} & \textcolor{blue}{Editoval Jan NOVÁK}\\
    \textcolor{red}{Verze} & \textcolor{red}{version} & \textcolor{red}{Verze 1.0}\\
    \textcolor{red}{Místo vydání} & \textcolor{red}{address} & \textcolor{red}{Brno}\\
    \textcolor{red}{Vydavatel} & \textcolor{red}{publisher} & \textcolor{red}{Fakulta informačních technologií VUT v Brně}\\
    \textcolor{blue}{Den} & \textcolor{blue}{day} & \textcolor{blue}{12}\\
    \textcolor{blue}{Měsíc vydání} & \textcolor{blue}{month} & \textcolor{blue}{2}\\
    Rok vydání & year & 2020\\
    \textcolor{blue}{Čas publikování} & \textcolor{blue}{time} & \textcolor{blue}{14:00}\\
    Revize & revised & Revidováno 12. 2. 2020\\
    Datum & citování & cited 2020-02-12\\
    \textcolor{magenta}{Identifikátor digitálního objektu} & \textcolor{magenta}{doi} & \textcolor{magenta}{10.1000/BC1.0}\\
    Standardní číslo & issn & 1234-5678\\
    \textcolor{red}{Poznámky} & \textcolor{red}{note} & \textcolor{red}{Toto je zcela vymyšlená citace}\\
    Dostupnost a přístup & url & https://merlin.fit.vutbr.cz
\end{tabularx}

\bigskip

\noindent \textbf{Bibliografická citace}

\medskip

\noindent \textsc{Nováková}, J. \textit{Web tvorby prací} [online]. Editoval Jan NOVÁK. Verze 1.0. Brno: Fakulta informačních technologií VUT v Brně, 2. března 1998 14:10. Revidováno 12. 2. 2020 [cit. 2020-02-12]. DOI: 10.1000/BC1.0. ISSN 1234-5678. Toto je zcela vymyšlená citace. Dostupné z: \url{https://merlin.fit.vutbr.cz}.

% Pro kompilaci po částech (viz projekt.tex) nutno odkomentovat
%\end{document}
