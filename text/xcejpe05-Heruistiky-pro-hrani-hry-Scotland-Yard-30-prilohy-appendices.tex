% Tento soubor nahraďte vlastním souborem s přílohami (nadpisy níže jsou pouze pro příklad)

% Pro kompilaci po částech (viz projekt.tex), nutno odkomentovat a upravit
%\documentclass[../projekt.tex]{subfiles}
%\begin{document}

% Umístění obsahu paměťového média do příloh je vhodné konzultovat s vedoucím
\chapter{Obsah přiloženého paměťového média}

\begin{itemize}
    \item \texttt{src/} -- Složka se zdrojovými kódy aplikace.
    \item \texttt{text/} -- Složka s~zdrojovými kódy textu práce v~jazyce \LaTeX.
    \item \texttt{trained\char`_policies\char`_dqn/} -- Složka s~natrénovaným modelem DQN.
    \item \texttt{trained\char`_policies\char`_ppo/} -- Složka s~natrénovaným modelem PPO
    \item \texttt{thesis.pdf} -- Soubor s~textem práce.
    \item \texttt{thesis\char`_print.pdf} -- Soubor s~textem práce pro tisk.
    \item \texttt{README.md} -- README soubor pro tuto práci.
    \item \texttt{simulations/} -- Složka s~výsledky simulací.
    \item \texttt{arial.ttf} -- Písmo pro grafické rozhraní
    \item \texttt{main.py} -- Soubor, jejož spuštěním se spustí grafické rozhraní hry Scotland Yard.
    \item \texttt{TrainerDQN.py} -- Soubor obstarávající trénování modelu DQN.
    \item \texttt{TrainerPPO.py} -- Soubor obstarávající trénování modelu PPO.
    \item \texttt{tune\char`_dqn.py} -- Tune byl použit pro hledání nejlepších hyperparametrů pro model DQN.
    \item \texttt{tune\char`_ppo.py} -- Tune byl použit pro hledání nejlepších hyperparametrů pro model PPO.
    \item \texttt{requirements.txt} -- Soubor s~python balíčky potřebnými pro spuštění aplikace.
\end{itemize}

%\chapter{Manuál}

%\chapter{Konfigurační soubor}

%\chapter{RelaxNG Schéma konfiguračního souboru}

%\chapter{Plakát}





% Pro kompilaci po částech (viz projekt.tex) nutno odkomentovat
%\end{document}
