% Tento soubor nahraďte vlastním souborem s přílohami (nadpisy níže jsou pouze pro příklad)

% Pro kompilaci po částech (viz projekt.tex), nutno odkomentovat a upravit
%\documentclass[../projekt.tex]{subfiles}
%\begin{document}

% Umístění obsahu paměťového média do příloh je vhodné konzultovat s vedoucím
%\chapter{Obsah přiloženého paměťového média}

%\chapter{Manuál}

%\chapter{Konfigurační soubor}

%\chapter{RelaxNG Schéma konfiguračního souboru}

%\chapter{Plakát}


\chapter{Checklist} 
\label{checklist}
Tento checklist byl převzat ze šablony pro kvalifikační práce, která je k dispozici na blogu prof. Herouta \cite{Herout}, který s laskavým dovolením využil nápadu dr. Szökeho%
\footnote{\url{http://blog.igor.szoke.cz/2017/04/predstartovni-priprava-letu-neni.html}}. 

Velká bezpečnost letecké dopravy stojí z části na tom, že lidé kolem letadel mají \textbf{checklisty} na úplně každý, třeba rutinní a dobře zažitý, postup. Jako pilot strpí to, že bude trochu za blbce a opravdu tužtičkou do seznamu úkonů odškrtá dokonale zvládnuté akce, vytiskněte si a odškrtejte před odevzdáním diplomky i vy tento checklist a vyhněte se tak častým chybám, které by mohly mít až fatální následky na výsledné hodnocení Vaší práce.

\subsubsection*{Struktura}
\begin{checklist}
	\item Už ze samotných názvů a struktury kapitol je patrné, že bylo splněno zadání.
	\item V textu se nevyskytuje kapitola, která by měla méně než čtyři strany (kromě úvodu a závěru). Pokud ano, radil(a) jsem se o tom s vedoucím a ten to schválil.
\end{checklist}

\subsubsection*{Obrázky a grafy}
\begin{checklist}
	\item Všechny obrázky a tabulky byly zkontrolovány a jsou poblíž místa, odkud jsou z textu odkazovány, takže nebude problém je najít.
	\item Všechny obrázky a tabulky mají takový popisek, že celý obrázek dává smysl sám o~sobě, bez čtení dalšího textu. Vůbec nevadí, když má popisek několik řádků.
	\item Pokud je obrázek převzatý, tak je to v popisku zmíněno: \uv{Převzato z [X].}
	\item Písmenka ve všech obrázcích používají font podobné velikosti, jako je okolní text (ani výrazně větší, ani výrazně menší).
	\item Grafy a schémata jsou vektorově (tj. v PDF).
	\item Snímky obrazovky nepoužívají ztrátovou kompresi (jsou v PNG).
	\item Všechny obrázky jsou odkázány z textu.
	\item Grafy mají popsané osy (název osy, jednotky, hodnoty) a podle potřeby mřížku.
\end{checklist}

\subsubsection*{Rovnice}
\begin{checklist}
	\item Identifikátory a jejich indexy v rovnicích jsou jednopísmenné (kromě nečastých zvláštních případů jako $t_\mathrm{max}$).
	\item Rovnice jsou číslovány.
	\item Za (nebo vzácně před) rovnicí jsou vysvětleny všechny proměnné a funkce, které zatím vysvětleny nebyly.
\end{checklist}

\subsubsection*{Citace}
\begin{checklist}
    \item \textbf{Všechny použité zdroje jsou citovány.}
	\item Adresy URL odkazující na služby, projekty, zdroje, github apod. jsou odkazovány pomocí \verb|\footnote{\url{...}}|.
    \item Všechny citace používají správné typy.
	\item Citace mají autora, název, vydavatele (název konference), rok vydání.  Když některá nemá, je to dobře zdůvodněný zvláštní případ a vedoucí to odsouhlasil.
	\item Je-li ve zdrojových textech programu něco převzaté, je to tam řádně citováno v souladu s licencí.
	\item Je-li podstatná část zdrojových textů programu převzatá, je toto zmíněno v textu práce a je citován zdroj.
\end{checklist}

\subsubsection*{Typografie}
\begin{checklist}
	\item Žádný řádek nepřetéká přes pravý okraj.
	\item Na konci řádku nikde není jednopísmenná předložka (spraví to nedělitelná mezera $\sim$).
	\item Číslo obrázku, tabulky, rovnice, citace není nikde první na novém řádku (spraví to nedělitelná mezera $\sim$).
	\item Před číselným odkazem na poznámku pod čarou nikde není mezera (to jest vždy takto\footnote{příklad poznámky pod čarou}, nikoliv takto \footnote{jiný příklad poznámky pod čarou}).
\end{checklist}

\subsubsection*{Jazyk}
\begin{checklist}
    \item Použil jsem kontrolu pravopisu a v textu nikde nejsou překlepy.
	\item Nechal jsem si text přečíst od (alespoň) jednoho dalšího člověka, který umí dobře česky / anglicky / slovensky.
	\item V práci psané česky nebo slovensky abstrakt zkontroloval někdo, kdo umí opravdu dobře anglicky.
	\item V textu se nikde nepoužívá druhá mluvnická osoba (vy/ty).
	\item Když se v textu vyskytuje první mluvnická osoba (já, my), vždy se popisuje subjektivní záležitost (\textit{rozhodl jsem se}, \textit{navrhl jsem}, \textit{zaměřil jsem se na}, \textit{zjistil jsem} apod.).
	\item V textu se nikde nepoužívají hovorové výrazy.
	\item V českém či slovenském textu se zbytečně nepoužívají anglické výrazy, které mají ustálené české překlady. Např. slovo \textit{defaultní} se nahradí např. slovem \textit{implicitní} nebo \textit{výchozí}.
\end{checklist}

\subsubsection*{Výsledek na datovém médiu, tj. software}
\begin{checklist}
	\item Mám připravené nepřepisovatelné datové médium 
      \begin{itemize}
	  		\item CD-R,
            \item DVD-R,
            \item DVD+R ve formátu ISO9660 (s rozšířením RockRidge a/nebo Jolliet) nebo UDF,
            \item paměťová karta SD (Secure Digital) ve formátu FAT32 nebo exFAT s nastavenou ochranou proti přepisu.
      \end{itemize}
	\item Pokud je výsledek online (služba, aplikace, \dots), URL je viditelně v úvodu a závěru, aby bylo jasné, kde výsledek hledat.
	\item Na médiu nechybí povinné: 
    	\begin{itemize}
    		\item zdrojové kódy (např. Matlab, C/C++, Python, \dots)
            \item knihovny potřebné pro překlad,
            \item přeložené řešení,
            \item PDF s technickou zprávou (je-li pro tisk 2. verze, tak obě),
            \item zdrojový kód zprávy (\LaTeX), 
    	\end{itemize}
        a případně volitelně po dohodě s vedoucím práce
		\begin{itemize}
			\item relevantní (např. testovací) data, 
            \item demonstrační video,
            \item PDF plakátku,
            \item \dots
		\end{itemize}        
	\item Zdrojové kódy jsou refaktorovány, komentovány a označeny hlavičkou s autorstvím, takže se v nich snadno vyzná i někdo další, než sám autor.
    \item Jakákoliv převzatá část zdrojového kódu je řádně citována -- tedy označena úvodním a v případě převzetí více řádků i ukončovacím komentářem. Komentář obsahuje vše, co vyžaduje licence uvedená na webu (vždy je nutné se ji pokusit najít -- např. Stack Overflow\footnote{\url{https://stackoverflow.blog/2009/06/25/attribution-required/}} má striktní pravidla pro citace).
\end{checklist}

\subsubsection*{Odevzdání}

\begin{checklist}
\item Chci práci (na max. 3 roky) utajit? Pokud ano, nejpozději měsíc před termínem odevzdání práce si podám žádost (v IS), ke které přiložím případné stanovisko firmy, jejíž duševní vlastnictví je třeba chránit.
\item Mám splněný minimální počet normostran textu (lze spočítat pomocí Makefile a~odhadem přičíst obrázky). Pokud jsem těsně pod minimem, konzultoval(a) jsem to s~vedoucím.
\item Pokud chci tisknout oboustranně, konzultoval(a) jsem to s~vedoucím a mám správně nastavenou šablonu. Kapitoly začínají na liché stránce.
\item Technickou zprávu mám v deskách z knihařství (min. 1 výtisk, při utajení oba).
\item Za titulním listem práce je zadání (tzn. mám jej stažené z IS a vložené do šablony).
\item V IS jsou abstrakty a klíčová slova.
  \begin{itemize}
    \item V abstraktu a klíčových slovech v IS nejsou zkopírované vlnky pro nezlomitelné mezery.
  \end{itemize}      
\item V IS je PDF práce (s klikatelnými odkazy).
\item Oba výtisky práce jsou podepsané.
\item V jednom (při utajení obou) výtisku práce je paměťové médium, na kterém je fixkou napsaný login (fixku na CD lze zapůjčit v knihovně, na Studijním oddělení nebo až při odevzdání).
\end{checklist}


% Pro kompilaci po částech (viz projekt.tex) nutno odkomentovat
%\end{document}
